\documentclass[12pt]{article}
\usepackage[breaklinks=true]{hyperref}
\usepackage{color}
\usepackage{amsmath,amssymb,amsthm}
\usepackage{natbib}
\usepackage{array}
\usepackage{booktabs, multicol, multirow}
\usepackage[nohead]{geometry}
\usepackage[singlespacing]{setspace}
\usepackage[bottom]{footmisc}
\usepackage{floatrow}
\usepackage{float,graphicx}
\usepackage{caption}
\usepackage{indentfirst}


\newtheorem{theorem}{Theorem}[section]
\newtheorem{lemma}[theorem]{Lemma}
\newtheorem{assumption}{Assumption}

\newcommand{\beq}{\begin{equation}}
\newcommand{\eeq}{\end{equation}}


\newcommand{\todo}[1]{{\color{red}{TO DO: \sc #1}}}

\newcommand{\reals}{\mathbb{R}}
\newcommand{\integers}{\mathbb{Z}}
\newcommand{\naturals}{\mathbb{N}}
\newcommand{\rationals}{\mathbb{Q}}

\newcommand{\ind}{\mathbb{I}} % Indicator function
\newcommand{\pr}{\mathbb{P}} % Generic probability
\newcommand{\ex}{\mathbb{E}} % Generic expectation
\newcommand{\var}{\textrm{Var}}
\newcommand{\cov}{\textrm{Cov}}

\newcommand{\normal}{N} % for normal distribution (can probably skip this)
\newcommand{\eps}{\varepsilon}
\newcommand\independent{\protect\mathpalette{\protect\independenT}{\perp}}
\def\independenT#1#2{\mathrel{\rlap{$#1#2$}\mkern2mu{#1#2}}}
\newcommand{\argmax}{\textrm{argmax}}
\newcommand{\argmin}{\textrm{argmin}}
\renewcommand{\baselinestretch}{1.5}

\title{Notes for categorical stochastic ordering test}
\author{}
\date{Draft \today}
\begin{document}
\maketitle


\section{Notation}

\todo{reconcile notation. Something about the way I'm denoting treatment is off! The null should be that $Y$ is equal in distribution across treatment groups.}

We have $n$ experimental units indexed $i=1, \ldots, n$.
Each unit receives a categorical treatment variable $X_i$ with levels $\{1, \ldots, k\}$.
Treatment is an \emph{ordered} categorical variable.
For example, treatment levels may be increasing doses of a drug.

Units have a vector of responses or outcomes $Y_i = (Y_{i1}, \ldots, Y_{iV})$.
Later on, we will describe the values that each variable $Y_{\cdot v}$ can take, $v = 1, \ldots, V$.

The null hypothesis is no effect of treatment:

$$H_0: X_1 \stackrel{d}{=} X_2 \stackrel{d}{=} \ldots \stackrel{d}{=} X_k.$$

The alternative is a stochastic ordering:

$$H_1: X_1 \stackrel{d}{\leq} X_2 \stackrel{d}{\leq} \ldots \stackrel{d}{leq} X_k,$$

where at least one inequality is strict.
The alternative hypothesis implies that the cumulative distribution functions of each treatment
are decreasing:

$$F_1(x) \geq F_2(x) \geq \ldots \geq F_k(x).$$


\section{Methods}

The alternative hypothesis implies the following relations hold:

\begin{align*}
F_1(x) &\geq F_{2 \cdots k}(x) \\
F_{12}(x) &\geq F_{3 \cdots k}(x) \\
&\vdots \\
F_{1 \cdots k-1}(x) &\geq F_{k}(x)
\end{align*}
where $F_{i_1 \cdots i_j}$ denotes the pooled distribution of individuals with $X_i \in \{i_1, \ldots, i_j\}$.

This suggests the following permutation testing procedure:

\begin{enumerate}
\item Permute the treatment vector $X$
\item Compute a test statistic for each of $V$ variables and the $k-1$ partitions of into two ordered treatment groups: 
$\{1\}$ against $\{2, \ldots, k\}$,
$\{1, 2\}$ against $\{3, \ldots, k\}$,
up to $\{1, 2, \ldots, k-1\}$ against $\{k\}$.
\item Use NPC to combine the $V \times (k-1)$ partial tests.
\end{enumerate}


\end{document}